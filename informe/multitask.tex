\section{Mecanismos de multitarea}
\label{sec::multitask}

El prop\'osito del Sistema Operativo en si es servir como capa de abstraci\'on de la
computadora para los procesos de nivel de usuario, y permitir adem\'as la multiplexaci\'on
de estos recursos entre multiples instancias de programas de usuario en ejecuci\'on. Lo que
buscamos en este Trabajo Pr\'actico fue proveer una capa b\'asica de interfaz para las tareas
de usuario. En este apartado describimos el mecanismo de multitarea provisto.

La implementaci\'on sigue la l\'inea estandar de cambios de contexto por hardware. Para esto
utilizamos fuertemente la estructura de TSS (Task State Segment) del procesador, lo cual
simplifica mantener al procesador en un estado consistente seg\'un la tarea en ejecuci\'on.
No incluimos una explicaci\'on detallada del uso de esta estructura pues es parte del programa
de la materia.

El tiempo de uso de CPU de cada proceso es un \textit{quantum} fijo. El scheduler entonces provee
una serie de funciones para manejar las tareas en modo \textit{round-robin}: El orden de ejecuci\'on
de las tareas es circular y no existen diferencias de prioridad de tareas. Para tener una noci\'on
de tiempo asociada a cada uno de estos procesos, utilizamos el PIT (Programmable Interrupt Timer).

El PIT es un reloj de cuarzo que env\'ia cada cierto intervalo de tiempo una interrupcci\'on por la l\'inea
1 del chip del Programmable Interrupt Controller. Esta interrupcci\'on nos sirve entonces como medida de paso
del tiempo. Para ajustarla a una cantidad determinada de milisegundos, lo configuramos mediante el uso del
puerto de comando $0x43$ y enviando la nueva frecuencia en dos bytes al puerto $0x40$ como esta detallado en 
\texttt{timer.c}. De esta manera tenemos manejo de cuanto tiempo pertimos correr a cada proceso.

Si bien no existen prioridades para los procesos, un proceso puede decidir dejar el procesador voluntariamente
mediante la llamada a sistema \texttt{sleep}. Esta es tambi\'en la \'unica manera que el procesador puede liberar
la computadora si se encuentra en modo kernel. Dado que el scheduler no puede retirar una tarea si esta se encuentra
en modo kernel, este es no preemteable. Sin embargo, si es posible interrumpir al kernel, como veremos posteriormente
cuando se explique la implementaci\'on del driver de terminal.

Para crear y manejar tareas se utiliza un esquema similar a las llamadas de Sistema Operativo cl\'asicas de UNIX y POSIX, \texttt{fork},
\texttt{exec}, \texttt{wait} y \texttt{exit}.

\begin{itemize}
	\item \texttt{fork} crea un duplicado del proceso actual, con el mismo mapa de memoria, archivos abiertos y directorio de trabajo.
	La relaci\'on entre el nuevo proceso y el proceso del cual se obtuvo es de padre e hijo: en particular como veremos posteriormente
	un proceso puede esperar por uno de sus hijos dado su n\'umero identificador de proceso (PID).
	\item \texttt{exec} sobreescribe la im\'agen del proceso con la im\'agen de proceso obtenida del archivo en disco duro pasado por
	par\'ametro, efectivamente entonces poniendo a ejecutar un proceso con una nueva secci\'on de texto y datos. \texttt{exec} cierra
	todos los archivos abiertos por esta nueva imagen exceptuando entrada y salida estandar. La relaci\'on padre e hijo se mantiene.
	\item \texttt{wait} bloquea un proceso hasta que uno de sus hijos (dado por el n\'umero PID pasado por par\'ametro) llama a la
	system call \texttt{exit}. Esto se usa por ejemplo en el shell para permitir la ejecuci\'on de una tarea y luego volver el control
	a la consola.
	\item \texttt{exit} Termina la ejecuci\'on del proceso y libera sus recursos. Adicionalmente, se desbloquea su padre si lo estaba
	esperando. Cuando un proceso hace \texttt{exit} sus hijos son reasignados a su padre en caso de tenerlo (se asume de todos modos que
	el proceso inicial \texttt{init} nunca hace exit).
\end{itemize}

La implementaci\'on, si bien en funcionalidad es cruda, demuestra como se podr\'ia estructurar este mecanismo para permitir multitarea.

Adicionalmente, se implement\'o como experimento un sistema b\'asico de manejo de se\~nales entre procesos. Las mismas se manejan solamente
en modo usuario y se dispone de un subconjunto de se\~nales: SIGSTOP, SIGCONT, SIGKILL y SIGSTOP. El kernel anota si las se\~nales fueron
recibidas en modo kernel y registra esto para manejarlas apenas retorna a modo usuario. Se permite adem\'as registrar handlers particulares
para se\~nales excepto para SIGSTOP SIGCONT y SIGKILL. 
Para que todos las funciones manejadoras de se\~nales se ejecuten en modo usuario, los handlers por default para cada una de estas se\~nales
se mantienen en una p\'agina especial de kernel que se mapea con permisos de usuario. Para bloquear un proceso ante la recepci\'on de
SIGSTOP, se implementa una syscall especial denominada \texttt{do\_coma} que bloquea permanentemente a un proceso. Esto solo se deshace ante
la recepci\'on de la se\~nal SIGCONT, cuyo handler no realiza acci\'on alguna. 

Cada proceso tiene asociada una estructura, adem\'as de la TSS, que controla que archivos tiene abiertos, su directorio
de trabajo actual, una lista enlazada (que casualmente es la implementaci\'on de lista enlazada intrusiva que utiliza el kernel de Linux)
de sus procesos hijos, su n\'umero de identificaci\'on, un puntero a la estructura de su padre (o NULL si es el proceso init).
