\appendix
\section{Como correr el Sistema Operativo}

Para correr el sistema operativo se necesita de una distribuci\'on de Linux
que cuente con \texttt{gnuutils}, en particular GCC y LD, \textit{nasm} y \texttt{bash}. 
Adem\'as, en particular se necesita 

\begin{itemize}
	\item Bochs 2.4.6 o m\'as moderno.
	\item La suite de utilidades \texttt{e2tools} para copiar archivos a la
	imagen de diskette. Se puede conseguir usando el \textit{package manager}
	de la distribuci\'on de Linux utilizada. En el caso de Ubuntu, se puede
	usar el comando \texttt{sudo apt-get install e2tools}.
	\item La suite de utilidades de particiones \texttt{mkfs}, en particular
	\texttt{mkfs.minix}. Para instalarlo puede utilizarse el comando
	\texttt{sudo apt-get install util-linux}.
\end{itemize}

Tambi\'en, dado que uno de los pasos requiere configurar y montar un \textit{loopback
device}, se necesitan privilegios de sudo para ejecutar ese script. Si bien se ha
tenido cuidado de no conflictuar ningun nombre, se avisa que es posible que se produzcan
problemas en este paso (por lo cual se recomienda leer el script \texttt{linkage/build\_image.sh} 
antes de ejecutar el siguiente comando).

Correr la simulaci\'on dadas las dependencias citadas anteriormente consiste en situarse en el directorio
m\'as arriba y ejecutar:

\begin{verbatim}
	make run
\end{verbatim}

El Makefile asume que los binarios de Bochs (en particular \texttt{bximage} y \texttt{bochs}) 
se encuentra en la carpeta \texttt{\textasciitilde /bochs/bin}
siendo \texttt{\textasciitilde} la carpeta home del usuario actual. Esto sin embargo se puede modificar
utilizando el prefijo \texttt{BOCHSDIR} al ejecutar \texttt{make} de la siguiente manera:

\begin{verbatim}
	make BOCHSDIR=path/a/bochs/bin run
\end{verbatim}
